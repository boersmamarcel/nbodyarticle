% !TEX TS-program = pdflatex
% !TEX encoding = UTF-8 Unicode

% This is a simple template for a LaTeX document using the "article" class.
% See "book", "report", "letter" for other types of document.

\documentclass[11pt]{article} % use larger type; default would be 10pt

\usepackage[utf8]{inputenc} % set input encoding (not needed with XeLaTeX)

%%% Examples of Article customizations
% These packages are optional, depending whether you want the features they provide.
% See the LaTeX Companion or other references for full information.

%%% PAGE DIMENSIONS
\usepackage{geometry} % to change the page dimensions
\geometry{a4paper} % or letterpaper (US) or a5paper or....
% \geometry{margin=2in} % for example, change the margins to 2 inches all round
% \geometry{landscape} % set up the page for landscape
%   read geometry.pdf for detailed page layout information

\usepackage{graphicx} % support the \includegraphics command and options

% \usepackage[parfill]{parskip} % Activate to begin paragraphs with an empty line rather than an indent

%%% PACKAGES
\usepackage{booktabs} % for much better looking tables
\usepackage{array} % for better arrays (eg matrices) in maths
\usepackage{paralist} % very flexible & customisable lists (eg. enumerate/itemize, etc.)
\usepackage{verbatim} % adds environment for commenting out blocks of text & for better verbatim
\usepackage{subfig} % make it possible to include more than one captioned figure/table in a single float
\usepackage{amsmath} 
% These packages are all incorporated in the memoir class to one degree or another...

%%% HEADERS & FOOTERS
\usepackage{fancyhdr} % This should be set AFTER setting up the page geometry
\pagestyle{fancy} % options: empty , plain , fancy
\renewcommand{\headrulewidth}{0pt} % customise the layout...
\lhead{}\chead{}\rhead{}
\lfoot{}\cfoot{\thepage}\rfoot{}

%%% SECTION TITLE APPEARANCE
\usepackage{sectsty}
\allsectionsfont{\sffamily\mdseries\upshape} % (See the fntguide.pdf for font help)
% (This matches ConTeXt defaults)

%%% ToC (table of contents) APPEARANCE
\usepackage[nottoc,notlof,notlot]{tocbibind} % Put the bibliography in the ToC
\usepackage[titles,subfigure]{tocloft} % Alter the style of the Table of Contents
\renewcommand{\cftsecfont}{\rmfamily\mdseries\upshape}
\renewcommand{\cftsecpagefont}{\rmfamily\mdseries\upshape} % No bold!

%%% END Article customizations

%%% The "real" document content comes below...

\title{N-Body problem}
\author{Marcel Boersma\\Shabaz Sultan}
%\date{} % Activate to display a given date or no date (if empty),
         % otherwise the current date is printed 

\begin{document}
\maketitle

\section{Introduction}
Commonly used technology in everyday life is supported by satellites, e.g. GPS in car navigation, Internet and pictures from Google Maps. 
And with increasingly more satellites orbiting the earth it is important that we know that a new satellite will not crash into another.
Or, in more general terms we need to know how objects orbiting other objects work such that we can launch more satellites.

Although, it is not a new problem, because we have been trying to analyze how planets orbit each other in our own solar system for quite while, there is still room for improvement.
The analytical solution for orbiting planets are only purposed for 2-body system, while in reality we have to cope with N-body systems. Approximation methods can be used only calculating is intractable by hand. Hence, the use of computers is promising for solving N-body systems. However, using computers and approximation methods comes with different types of problems, i.e. how accurate is my approximation method as well as the physical limitations of computers in terms of floating point operations.

This research is aims to provide insight in the approximation methods used for 2-body systems, i.e. how accurate is the simulation. 
In section~\ref{sec:literature} the background of the analytical models used are explained and section~\ref{sec:methodology} introduces the algorithms as well a methods for benchmarking our approximation methods. The results are shown in section~\ref{sec:results} and section~\ref{sec:conclusion} our conclusion is discussed.

\section{Background}
\label{sec:literature}
In pursuance of calculating the trajectory of a body it is necessary to know which forces interact with our body. Newton's seconds law, i.e. the rate of change is equal to the net force on a object, is used to describe the force on our object. And Newton's third law, i.e. action = - reaction, is used. 

\subsection{Newton's laws}
Newton's second law describes that the force on an object is equal to the change of momentum in time. Because we only consider constant mass systems this is equal to the mass times the acceleration of an object. Thus, if the object is accelerating there is an force on our object present. Knowing that the only possible object acting on our object are the other planets in the system we can modify the second law to derive the law of universal gravitation. Let the following equation be the force $F_p$ on object $p$ by object $c$ with mass $m_p,m_c$ respectively.   
\begin{equation}
    \label{eq:newtongravity}
 F_p = G\frac{m_pm_c}{r^2}
\end{equation}
With $r$ as the euclidean distance between object $p,c$. Combining Newton's second law with the law of universal gravitation gives the acceleration $a_p$ of object $p$
\begin{equation}
    \label{eq:newtongravity}
	a_p = G\frac{m_pm_c}{ m_pr^2}
\end{equation}.
The third law states that if there is a force $F_1$ then there also exists a force $F_2$ such that $F_2 = - F_1$. 

In the next section a simple 2-body system is introduced for thorough understanding the approximation methods used in later sections.


\section{Methodology}
\label{sec:methodology}
To understand the consequences of our approximation methods we first need to understand the analytical solution of a 2-body system. Next, we assume a simple two body system with an analytical solution and we will apply Newton's law to obtain the trajectory of our objects. Remark that this is special simple case designed to provided better understanding of orbiting objects, this is required to understand the effects of discrete approximation methods in our simulations. \\
\indent Assume we have a 2 body system called Pluto-Charon system. Figure~\ref{fig:plutocharon} illustrates this system at a fixed time point $t$
\begin{figure*}
	\label{fig:plutocharon}
\end{figure*}
with both $x,y$ coordinates in our 2D space.

% Assume that we have an initial velocity vector $\overrightarrow{v_t}$ for the body t with speed in both x and y direction.
% \begin{equation}
% 	\overrightarrow{v_t} = \begin{bmatrix}
% 								x_t \\
% 								y_t
% 							\end{bmatrix}
% \end{equation}
% For example we have a two body system with $p,c$ with an initial velocity vector, respectively, then
% \begin{equation}
% 	v_p=\begin{bmatrix} v_x^p \\ v_y^p \end{bmatrix}, v_c=\begin{bmatrix} v_x^c \\ v_y^c \end{bmatrix}
% \end{equation}
We assume that the whole system has a constant velocity, in our case we assume zero velocity, hence the next equation must be equal to zero
\begin{equation}
	V = \frac{P}{M}	
\end{equation}
with $P$ as the total momentum and $M$ as the total mass of the system. We know that the total momentum is the sum of all forces acted up on and the mass is a constant. Therefore, our zero velocity implies that the total momentum must be equal to zero. Or more formally
\begin{equation}
	P = m_p\overrightarrow{v_p} + m_c\overrightarrow{v_c} = 0.
\end{equation}
As illustrated in figure~\ref{XXXX}(simple two body system with vectors drawn) we can see that each body has several forces acted upon. Our objective is the determine the path of body $p,c$ and as you can see we have a force vector $F_x$ pulling the body $X$ to the center and also our velocity vector $v_x$. The product of those vectors gives us the directional vector of our body $X$. Due to the fact that the two bodies are the only objects in the system acting on eachother we can determine the radius and period of our movement around a (virtual) center of mass. We know for a fact that the two bodies must be the opposites of eachother such that the total momentum is equal to zero. The center of mass can be calculated in reference of one of our bodies.  (bary center)
\begin{equation}
	(m_p + m_c)\overrightarrow{r_{m}} = m_p\overrightarrow{r_p} + m_c\overrightarrow{r_c} 
\end{equation}
With $r_m$ being the vector to the center of mass, the we can rewrite this to calculate $r_m$
\begin{equation}
	\overrightarrow{r_m} = \frac{m_p\overrightarrow{r_p} + m_c\overrightarrow{r_c}}{(m_p + m_c)}
\end{equation}
and we will calculate the $\overrightarrow{r_m}$ with as reference point our body $C$, then the $\overrightarrow{r_c}=0$ because the distance between our body $C$ and $C$ is zero, and the $\overrightarrow{r_p}=R$ with $R$ as the total distance between the two bodies. Then $\overrightarrow{r_m}$ becomes
\begin{equation}
	\overrightarrow{r_m} = \frac{m_p}{M}\overrightarrow{R}
\end{equation}
force times direction (unit vector) see albana notes on slide
Calculate $a_i$ for each body i.


If we want to know the force of body $p$ on body $c$ we can use Newton's law of gravitation, see equation~\ref{eq:newtongravity}.  We can fill in the masses of bodies $p,c$ using the fact that $F_c = m_c a_c$ we obtain
\begin{equation}
    \begin{split}
     a_c m_c = G\frac{m_c m_p}{r^2} \\
     a_c = G\frac{m_p}{r^2} \\
    \end{split}
\end{equation}
Now that we have derived the acceleration in terms of the mass of $p$ and the distance between the bodies $r$, we can use this calculate the acceleration of body $c$. The change of position i.e. distance of body $c$ at time $\delta t$ is given by
\begin{equation}
    \frac{d^2}{dt^2} r_c = G \frac{m_p}{r^2}
\end{equation}
So the new position can be derived by taking the position at time $t=0$ plus the distance traveled during $\delta t$. This can be obtained by taking the double integral of the acceleration. This can be solved analytically when only considering two bodies. If you consider $n \geq 3$ bodies it becomes unsolvable analytically. 
\subsection{Discrete Approximation}
The N-body system for more than 2 bodies can not be solved analytically, hence we must use approximation methods to calculate the solution.
We want to obtain the new positions of our bodies at time $t+1$ by using Newton's second law. We can derive the next position by adding the distance traveled in time $\Delta t$ to the old position at $x_t$, i.e. $x_{t+1} = x_t + v_t*\Delta t$ wit $v_t$ as the speed at time $t$. Nonetheless, we do not know speed $v_t$ but this can be obtained by $v_t = v_t + a_t \Delta t$ and $a_t$ is known for our body. The next section describes an even more accurate approach for $x_t$.

\subsubsection{Taylor expansion}
If we want to approximate the values of, for example, formula $f(x)$ at point $a$ we can use a Taylor expansion defined as
\begin{equation}
    \label{eq:taylor}
    \sum^\infty_{n=0} \frac{f^{(n)}(a)}{n!}(x-a)^n
\end{equation}
when we sum to infinity the new function will be the same as the original function $f(x)$. In our case we want to approximate the formulas $x(t), v(t), a(t)$ at points near $t$ and we can use equation~\ref{eq:taylor} to describe $x(t)$ as with 
\begin{equation}
    x(t) = \sum^\infty_{n=0} \frac{x^{(n)}(t)}{n!}(x-t)^n
\end{equation}
however, we are dealing with limited precision and will only use a few terms of the summation such that $x(t)$ becomes
\begin{equation}
    x(t) =  \frac{x^{(0)}(t)}{0!}(x-t)^0 + \frac{x^{(1)}(t)}{1!}(x-t)^1 +\frac{x^{(2)}(t)}{2!}(x-t)^2 +\frac{x^{(3)}(t)}{3!}(x-t)^3
\end{equation}
notice that $x^{(1)}(t)$ is the derivative of $x(t)$ which is equal to the velocity $v(t)$ and $x^{(2)}$ is equal to the acceleration $a(t)$. But we can also formulate a Taylor expansion for $v(t)$ and $a(t)$
\begin{equation}
    \begin{split}
        v(t) =  \frac{v^{(0)}(t)}{0!}(x-t)^0 + \frac{v^{(1)}(t)}{1!}(x-t)^1 +\frac{v^{(2)}(t)}{2!}(x-t)^2\\
        v(t) = v(0) + a(t) + j(t)
    \end{split}
\end{equation}
\begin{equation}
    \begin{split}
        a(t) =  \frac{a^{(0)}(t)}{0!}(x-t)^0 + \frac{a^{(1)}(t)}{1!}(x-t)^1 \\
        a(t) =  a(0) + j(t) 
    \end{split}
\end{equation}
we can calculate $j(t)$ by taking the derivative of equation~\ref{eq:newtongravity} with respect to $a$. Thus obtaining more accurate results for our $x(t), v(t)$ and $a(t)$.
Also, error of the approximation can be calculated as follows:


\subsubsection{Sensitivity analysis}
because we use approximations we have sensitive systems etc....
perturbations in input variables and output variables. i.e. positions of other objects are incorrect.
Delta step influences sensitivity?....


\subsubsection{Verify our approximation}
To verify the accuracy of the simulation we can apply some fundamental conservation laws from physics to our simulation. We know based on empirical data that these laws seem to be true for the natural, so if our simulation is a faithful model of nature these laws should also be true for our simulation. If they do not, the amount our simulation fails to conserve certain values that are supposed to be conserved can be used as a metric for the error in our simulation.\\\\
\textbf{Conservation of Momentum}\\
Momentum is defined as the mass of an object multiplied with its velocity vector. To get the momentum of a system of n bodies you can calculate the momentum of each body and sum them all together.
\begin{equation}
\vec{p}_{total} = \sum_i m_i \times \vec{v_i}
\end{equation}
As long as two interacting bodies experience equal but opposite forces the conservation of momentum tends to be pretty well preserved, even if those forces are otherwise wildly inaccurate. One the one hand this means that a fundamental law is enforced with any algorithm that ends up with the same force with a body to body interaction, just in opposite direction. This will be true for the n-body algorithms covered in this paper. On the other this means that the conservation of momentum is not a useful error metric to check the accuracy of an algorithm. Hence it is not used in this paper to compare the accuracy of the studied algorithms. The measurements have been done, but show no real difference, even when it is clear that a simulation is doing things that are unrealistic and inaccurate.
\textbf{Energy}\\
Constraints check after each simulation round
Possible checks for simulation correctness: law of momentum conservation, law of energy conservation, unintended orbit change



\subsection{Algorithms}
The n-body problem involves every body in a system interacting with every other body through gravity. This means that there are $\mathcal{O}(n^2)$ interactions that need to be calculated. Very broadly n-body simulators fall into two categories. The first category contains direct solvers, which simulate the gravitational interactions between all bodies directly, e.g. phiGRAPE \cite{Harfst2007357}. \\
Alternatively, there are solvers that use a tree to represent the set of bodies in a hierarchical manner. These algorithms have each of the bodies interact with the tree instead of the full set of n-bodies. This can lead to a computational complexity of $\mathcal{O}(n \log(n))$. The Barnes-Hut algorithm is a pioneering example of this class of n-body simulation algorithms \cite{barnes1986hierarchical}.\\
We have opted to go for a direct simulator, because the implementation of such algorithms tends to be less complex. The important part in these simulations is the determination of the force each of the bodies experiences. In direct solvers, this is done the same for each body to body interactions and the total force on a particular body is a simple summation of the forces it experiences from every other body. \\
To determine the force between two bodies due to gravity is a second order differential equation needs to be solved. Thus numerical integrators are the key component of direct solvers.
\subsubsection{Integrator Algorithms}
\textbf{Euler's Method}\\
The integrators need to solve a second order differential equation. Based on calculating the force at a current point in time acceleration for a body can be determined. Based on this acceleration the position can be updated. The most straightforward way to do this is to apply Euler's method. Based on an acceleration and a stepsize $\Delta t$ it linearly approximates the speed at time $t + \Delta t$ from the speed at time $t$. Using the speed it then updates the position at time $t+\Delta t$ by linearly extrapolating from position at time $t$.
\begin{equation}
    \begin{split}
    x(t+ \Delta t) = x(t) + v(t) \Delta t \\
    v(t+ \Delta t) = v(t) + a(t) \Delta t \\
    \end{split}
\end{equation}
This integrator is appealing for its simplicity, but because it is essentially a first order taylor expansion the expected error is proportional to $\Delta t^2$.
\subsection{Implementation}
attachment
\section{Experimental Results}
\label{sec:results}
results

\section{Conclusions}
\label{sec:conclusion}
\bibliographystyle{alpha}
\bibliography{article}
\end{document}

