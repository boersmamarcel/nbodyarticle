% !TEX TS-program = pdflatex
% !TEX encoding = UTF-8 Unicode

% This is a simple template for a LaTeX document using the "article" class.
% See "book", "report", "letter" for other types of document.

\documentclass[11pt]{article} % use larger type; default would be 10pt

\usepackage[utf8]{inputenc} % set input encoding (not needed with XeLaTeX)

%%% Examples of Article customizations
% These packages are optional, depending whether you want the features they provide.
% See the LaTeX Companion or other references for full information.

%%% PAGE DIMENSIONS
\usepackage{geometry} % to change the page dimensions
\geometry{a4paper} % or letterpaper (US) or a5paper or....
% \geometry{margin=2in} % for example, change the margins to 2 inches all round
% \geometry{landscape} % set up the page for landscape
%   read geometry.pdf for detailed page layout information

\usepackage{graphicx} % support the \includegraphics command and options

% \usepackage[parfill]{parskip} % Activate to begin paragraphs with an empty line rather than an indent

%%% PACKAGES
\usepackage{booktabs} % for much better looking tables
\usepackage{array} % for better arrays (eg matrices) in maths
\usepackage{paralist} % very flexible & customisable lists (eg. enumerate/itemize, etc.)
\usepackage{verbatim} % adds environment for commenting out blocks of text & for better verbatim
\usepackage{subfig} % make it possible to include more than one captioned figure/table in a single float
\usepackage{amsmath} 
% These packages are all incorporated in the memoir class to one degree or another...

%%% HEADERS & FOOTERS
\usepackage{fancyhdr} % This should be set AFTER setting up the page geometry
\pagestyle{fancy} % options: empty , plain , fancy
\renewcommand{\headrulewidth}{0pt} % customise the layout...
\lhead{}\chead{}\rhead{}
\lfoot{}\cfoot{\thepage}\rfoot{}

%%% SECTION TITLE APPEARANCE
\usepackage{sectsty}
\allsectionsfont{\sffamily\mdseries\upshape} % (See the fntguide.pdf for font help)
% (This matches ConTeXt defaults)

%%% ToC (table of contents) APPEARANCE
\usepackage[nottoc,notlof,notlot]{tocbibind} % Put the bibliography in the ToC
\usepackage[titles,subfigure]{tocloft} % Alter the style of the Table of Contents
\renewcommand{\cftsecfont}{\rmfamily\mdseries\upshape}
\renewcommand{\cftsecpagefont}{\rmfamily\mdseries\upshape} % No bold!

%%% END Article customizations

%%% The "real" document content comes below...

\title{N-Body problem}
\author{Marcel Boersma\\Shabaz Sultan}
%\date{} % Activate to display a given date or no date (if empty),
         % otherwise the current date is printed 

\begin{document}
\maketitle

\section{Introduction (phenomena)}
A world famous problem is the two body problem, most commonly known as the sun - earth problem.

\section{Literature Study}
background etc
\subsection{Newton's law}
\begin{equation}
 F = G\frac{m_1m_2}{r^2}
\end{equation}
With G as a gravitational constant and F as the force in Newton and $m_1,m_2$ the mass of the two systems in kg and $r$ is the distance between the center of the two bodies (see figure xx). 
Keplers problem? 
\subsection{Kepler}
Kepler problem, special case the Kepler orbit problem



\section{Methodology}
Assume that we have an initial velocity vector $\overrightarrow{v_t}$ for the body t with speed in both x and y direction.
\begin{equation}
	\overrightarrow{v_t} = \begin{bmatrix}
								x_t \\
								y_t
							\end{bmatrix}
\end{equation}
For example we have a two body system with $p,c$ with an initial velocity vector, respectively, then
\begin{equation}
	v_p=\begin{bmatrix} v_x^p \\ v_y^p \end{bmatrix}, v_c=\begin{bmatrix} v_x^c \\ v_y^c \end{bmatrix}
\end{equation}
We assume that the whole system has a constant velocity, in our case we assume zero velocity, hence the next equation must be equal to zero
\begin{equation}
	V = \frac{P}{M}	
\end{equation}
with $P$ as the total momentum and $M$ as the total mass of the system. We know that the total momentum is the sum of all forces acted up on and the mass is a constant. Therefore, our zero velocity implies that the total momentum must be equal to zero. Or more formally
\begin{equation}
	P = m_p\overrightarrow{v_p} + m_c\overrightarrow{v_c} = 0.
\end{equation}
As illustrated in figure~\ref{XXXX}(simple two body system with vectors drawn) we can see that each body has several forces acted upon. Our objective is the determine the path of body $p,c$ and as you can see we have a force vector $F_x$ pulling the body $X$ to the center and also our velocity vector $v_x$. The product of those vectors gives us the directional vector of our body $X$. Due to the fact that the two bodies are the only objects in the system acting on eachother we can determine the radius and period of our movement around a (virtual) center of mass. We know for a fact that the two bodies must be the opposites of eachother such that the total momentum is equal to zero. The center of mass can be calculated in reference of one of our bodies.  (bary center)
\begin{equation}
	(m_p + m_c)\overrightarrow{r_{m}} = m_p\overrightarrow{r_p} + m_c\overrightarrow{r_c} 
\end{equation}
With $r_m$ being the vector to the center of mass, the we can rewrite this to calculate $r_m$
\begin{equation}
	\overrightarrow{r_m} = \frac{m_p\overrightarrow{r_p} + m_c\overrightarrow{r_c}}{(m_p + m_c)}
\end{equation}
and we will calculate the $\overrightarrow{r_m}$ with as reference point our body $C$, then the $\overrightarrow{r_c}=0$ because the distance between our body $C$ and $C$ is zero, and the $\overrightarrow{r_p}=R$ with $R$ as the total distance between the two bodies. Then $\overrightarrow{r_m}$ becomes
\begin{equation}
	\overrightarrow{r_m} = \frac{m_p}{M}\overrightarrow{R}
\end{equation}
force times direction (unit vector) see albana notes on slide
Calculate $a_i$ for each body i.



\subsection{Discrete Approximation}
Approximation of Newton's law, Kepler orbit. Possible accuracy problems, e.g. sensitivity analysis.

\subsection{Algorithms}

\subsection{Implementation}
\section{Experimental Results}
\section{Conclusions}

\end{document}

